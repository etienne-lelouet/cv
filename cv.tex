\documentclass[11pt]{article}
\usepackage[a4paper, total={6in, 8in}, margin=10mm]{geometry}
\usepackage[T1]{fontenc}
\usepackage[utf8]{inputenc}
\usepackage[french]{babel}
\usepackage{xcolor}
\usepackage[none]{hyphenat}
\usepackage{setspace}
\usepackage{hyperref}

\pagenumbering{gobble}

\definecolor{headingblue}{HTML}{81ABD2}
\setlength{\parindent}{0em}

\newcommand{\resumesectiontitle}[1]{
    \noindent% prevent the box from being shifted
    \colorbox{headingblue}{% create a colored box
        \makebox[\textwidth][c]{% center the text on the page
            \textbf{#1}
        }
    }
}

\begin{document}
\sffamily
\begin{center} % just for vertical spacing and killing indent
    \begin{tabular*}{\textwidth}{@{}l@{\extracolsep{\fill}}r@{}}
        \textbf{Etienne LE LOUËT}  & 17/11/1996, Paris (nationalité française)\\
        Doctorant en informatique  &  26 rue de la Providence, 75013, Paris\\
        & etiennelelouet@outlook.com\\
        & \href{https://github.com/etienne-lelouet}{github.com/etienne-lelouet}
    \end{tabular*}
\end{center}

\textit{Actuellement doctorant en informatique, ma thèse porte sur l'optimisation des performances des protocoles de transport sécurisés utilisés par les différents acteurs du DNS.}

\vspace{1em}

\resumesectiontitle{Cursus}

\textbf{\underline{2019 - 2021}, Sorbonne Université :} Master d'informatique - Systèmes et Applications Réparties - Mention Bien

\vspace{1em}

\textbf{\underline{2018 - 2019}, Sorbonne Université :} Licence d'informatique en apprentissage DANT - Mention Bien

\vspace{1em}

\textbf{\underline{2016 - 2018}, CFA Insta :} BTS en informatique spécialisée dans le développement d'applications

\vspace{1em}

\resumesectiontitle{Expériences professionnelles}

\textbf{\underline{Mars 2021 - Aujourd'hui} - Thèse CIFRE - GANDI, Sorbonne Université (LIP6) } :
\begin{itemize}
\item Redaction d'un article portant sur les performances des différents transports sécurisés pour DNS
\item Rédaction d'une revue sur l'état de l'art des technologies de sécurisation de DNS
\end{itemize}

\vspace{1em}

\textbf{\underline{Septembre 2018 - Septembre 2019} - Développeur Full Stack en apprentissage - startup KMB Labs} :
\begin{itemize}
    \item Réalisation d'une application micro-services (Amazon Web Services  NodeJS, React et MongoDB).
\end{itemize}

\vspace{1em}

\textbf{\underline{Septembre 2016 - Septembre 2018} - Développeur Full Stack en apprentissage - startup SPOTLOOK} :
\begin{itemize}
    \item Développement web Fullstack (PHP/MySQL, HTML/CSS, JavaScript, VB.net)
\end{itemize}

\vspace{1em}

\resumesectiontitle{Enseignement}

\textbf{Chargé de TD :}
\begin{itemize}
\item \textbf{\underline{Premier Semestre 2022/2023 et 2023/2024} - Programmation Java avancée}
\item \textbf{\underline{Deuxième semestre 2021/2022} - Programmation Noyau}
\item \textbf{\underline{Premier semestre 2021/2022} - Programmation Système Concurrente et Répartie}
\end{itemize}

\vspace{1em}

\resumesectiontitle{Publications scientifiques}

\begin{itemize}
\item E. L. Louet, A. Blin, J. Sopena, A. Amamou and K. Haddadou, "Effects of secured DNS transport on resolver performance" 2023 IEEE Symposium on Computers and Communications (ISCC), Gammarth, Tunisia, doi: 10.1109/ISCC58397.2023.10217887.
\item E. L. Louet, A. Blin, J. Sopena, A. Amamou and K. Haddadou, "Effets de l'utilisation de transports sécurisés sur les performances d'un resolveur DNS" Conférence francophone d'informatique en Parallélisme, Architecture et Système (COMPAS), Annecy, France, 2023
\end{itemize}

\vspace{1em}

\resumesectiontitle{Projets informatiques personnels}

\begin{itemize}
    \item Auto-hébergement : hébergement de sites web, seedbox, serveur DNS (autoritaire et resolveur)
    \item Développement d’outils pour environnement de bureau Linux
\end{itemize}

\vspace{1em}

\resumesectiontitle{Compétences}

\begin{itemize}
\item Programmation système : APIs POSIX, Noyau Linux (pile TCP/IP, drivers), eBPF, OpenSSL, GNUTLS, LibUV
\item Autres langages : Python (pandas, matplotlib), PHP, NodeJS, Bash, design patterns objet (Java)
\item Outils: Git, linux-perf, docker, latex,bases de données relationnelles
\item Architecture des réseaux: pile TCP/IP, DNS, TLS, HTTP
\item Langues étrangères - anglais courant, espagnol intermédiaire
\end{itemize}

\vspace{1em}

\resumesectiontitle{Loisirs}

Cuisine, escalade, philosophie, musique expérimentale.

\end{document}
